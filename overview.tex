\section{Overview}
\subsection{Introduction}
\subsection{Disclaimer}
This audit does not give any warranties on the bug-free status of the given code, i.e., the evaluation result does not guarantee the nonexistence of any further findings of security issues. This audit report is intended to be used for discussion purposes only. We always recommend proceeding with several independent audits and a public bug bounty program to ensure the security of the project.
\subsection{Executive Summary}
\subsection{Security Opinion \& Code Quality}
\subsection{Findings Severity Breakdown}
The findings are classified under the following severity categories according to the impact and the likelihood of an attack.

\begin{center}
\begin{tabular}{||c p{9cm}||} 
 \hline
 Level & Description \\ [0.5ex] 
 \hline\hline
 \textcolor{red}{High} & Logical errors or implementation bugs that are easily exploited. In case of contracts such issues can lead to any kind of loss of funds.\\
 \hline
 \textcolor{purple}{Medium} & Issues that may break the intended logic, is a deviation from the specification or can lead to DoS attacks.\\
 \hline
 \textcolor{blue}{Low} & Issues harder to exploit (exploitable with low probability), issues that lead to poor performance, clumsy logic or seriously error-prone implementation.\\
 \hline
 \textcolor{green}{Informational} & Advisory comments and recommendations that could help make the codebase clearer, more readable and easier to maintain.\\
 \hline
\end{tabular}
\end{center}


