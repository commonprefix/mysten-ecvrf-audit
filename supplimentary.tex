\section{Supplementary Information}
\subsection{Adding Batch Verification Support}
The current ECVRF implementation in the fastcrypto library does not support batch verification. Batch verification can improve the efficiency of verifying multiple VRF proofs simultaneously. A recent paper by Badertscher et al.\cite{batch-ecvrf}, describes a batching technique based on randomized testing that can double verification speed when verifying batches of 1024 ECVRF proofs. The batched proofs do not need to pertain to the same user or the same input. To facilitate this, they change the representation of the proof, replaceing the $c$ component of the VRF proof with $U,V$ as calculated by the prover.


Such proofs use an alternate verification process, where the verifier derives $c$ from the challenge generation function and checks consistency via two verification equations (as opposed to deriving two values $U,V$). This is effectively a reordering of standard verification: there, we use $c,s$ to derive $U,V$ via the same equations and then use $U,V$ to check the derived value is equal to the one provided by the verifier. Because standard verification needs to use $U,V$ as hash inputs, it’s not possible to reduce the effort in the calculation. In alternate verification however, we only need a true/false output out of each equation. In fact, we are only interested in the logical AND of the two. This enables us to reduce effort in exchange of a small soundness error.   

We shift the terms so that each equation sums to zero. Due to the Schwartz–Zippel lemma, a randomized sum of such equations will almost always be non-zero if even one of the terms is non-zero (i.e of one of the equations is false). In computational terms, we only need to perform one (very long) multi-multiplication and some field operations. These optimizations can also be combined with the variant multi-multiplication functions provided by Dalek. The main practical cost to this approach is that $U,V$ are 32 bytes each instead of 16 bytes for $c$.

Batch-format proofs can be ``compressed'' to the standard form at negligible computational cost, if not verifying at the same time. Standard proofs can be made batch friendly, though the conversion cost is equivalent to a full verification.

We recommend exploring the addition of batch verification support to the ECVRF implementation in the fastcrypto library, following the approach described in the referenced paper. This could enhance the efficiency and performance of the library, particularly when verifying multiple VRF proofs at once.

\subsection{Secret Scalar and Nonce Derivation}
The ECVRF standard provides two separate notions of a prover’s secret: the secret key $SK$ and the (derived) secret scalar $x$. In general, the secret key is used in two places: first, to produce the public key via $x$ and second to set a prefix used in the nonce generation for proofs. There are two different design paths described in the RFC:
\begin{itemize}
\item Suites deriving from P-256 have a simple structure, where the key, the scalar and the prefix are the same value (modulo representation of the secret as a bytestring for the prefix). Suites deriving from P-256 use RFC Section~5.4.2.1 to derive nonces.

\item Suites deriving from ed25519 have a more complex structure, where the key is a bytestring hashed into a 64 byte string, half of which is used to derive the secret scalar and half of which is used to derive the nonce as per RFC Section~5.4.2.2.

\end{itemize}
The current implementation uses the P-256 design for the secret scalar and the ed25519 design for the nonce. This does not directly contradict the guidance in the RFC, as there is no specific dependence between the representation of $SK$ and the nonce derivation. Additionally, the hLen requirement of 64 mentioned in RFC Section~5.4.2.2 is met by the implemented instantiation. We do not believe there exists an issue with this design choice: domain separation is not impacted and the contribution of the secret key to the nonce is similar. 

We note that it seems that aligning fully with either of the two RFC designs is not trivial. One either needs to follow the more complex HMAC design from P-256 or follow  the scalar derivation from ed25519 which includes operations that are unneeded in ristretto. Part of the ed25519 scalar derivation performs bit operations with the intention of clearing the ed25519 cofactor which does not exist in the ristretto curve. As such, following ed25519 would involve unnecessary operations, otherwise, a degree of divergence may still be present.

