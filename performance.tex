\section{Performance Optimizations and Further Benchmarking}
In our analysis, we utilized the existing benchmark code for ECVRF, which was developed using the Criterion crate. Our focus was on optimizing the verification function, as it is a crucial component in a blockchain setting where it is performed multiple times by a large validator set.

By generating flamegraphs for the verification benchmark, we were able to identify parts of the code that took the most time. The two multiscalar multiplication operations were the most time-consuming, followed by compressing public keys.

To address these performance bottlenecks, we proposed using vartime multiscalar multiplication (L-02), which is faster than constant-time multiscalar multiplication, and caching public key compression (I-05). Our analysis indicates that implementing both L-02 and I-05 can improve performance by 27\%.

We also suggested additional caching techniques in I-04 and I-06. However, further work is required to concretely benchmark the improvements offered by these techniques and determine whether their implementation is warranted. While we strongly recommend adopting L-02 and I-05 due to their demonstrated benefits, more research is needed for I-04 and I-06. A careful trade-off must be considered between code readability and performance gains before deciding to implement these optimizations.